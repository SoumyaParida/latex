\thispagestyle{empty}
\vspace*{0.2cm}

\begin{center}
    \textbf{Zusammenfassung}
\end{center}

\vspace*{0.2cm}

\noindent 
Da die meisten Leuten an der TU deutsch als Muttersprache haben, empfiehlt es sich, das Abstract zus�tzlich auch in deutsch zu schreiben. Man kann es auch nur auf deutsch schreiben und anschlie�end einem Englisch-Muttersprachler zur �bersetzung geben. \\\\

\noindent
Jeden tag verwenden mehr und mehr Nutzer mobile Endger�te. Eine Vielfalt von mobilen Applikationen str�mt auf den Markt und Mobilfunkkunden ben�tigen schnellere Verbindungen mit h�heren Datenraten. In Anbetracht der steigenden Anforderungen der Endnutzer, m�ssen Netzwerkbetreiber ihre Systeme aufr�sten. \\\\

\noindent
Die meisten legacy Netzwerke werden von ihren Betreibern �berprovisioniert, um Lastspitzen bedienen zu k�nnen. Dies f�hrt zu exzessiven Kosten, da die durchschnittliche Auslastung der Netzwerke den Gro�teil der Zeit deutlich geringer ist. Daher wenden diese Betreiber sich cloud basierten Infrastrukturen, welche eine effizientere Steuerung der Netzwerkresourcen erm�glichen. Mobilfunkanbieter stellen ihre Netzwerke auf virtuelle Infrastrukturen um, mit denen sie ihr Netzwerk dynamisch den aktuellen Lastanforderungen anpassen k�nnen. Diese Arbeit konzentriert sich auf die Anforderungen des Kernnetzes. \\\\

\noindent
Es ist von Vorteil eine Mechanismus zur Skalierung von MMEs im Kernnetz zu haben, der f�r die Endnutzer nicht detektierbar ist. Bereits bestehende Ans�tze, wie S1-Flex und random load balancing, k�nnen das Kernnetz skalieren. Diese sind jedoch nicht transparent f�r den Endnutzer oder die eNodeB. Daher ist ein Lastausgleichsverfahren auf der S1-MME Protokollebene notwending, welcher den Lastausgleich transparent und dem 3GPP Standard so konform wie m�glich realisiert. Diese Arbeit befasst sich mit dem Design und der Implementierung eines S1-MME Loadbalancers, welcher eingehenden Mobilfunkverkehr �ber mehrere MMEs verteilen kann ohne die Affinit�t auf der operativen Ebene zu verlieren. \\\\
