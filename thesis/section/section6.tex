This chapter analyzes the measurements obtained from chapter 5 to identify hyper giants by studying the features of clustered hosting infrastructure such as number of links served by the cluster, number of IP addresses that belong to the cluster, number of BGP prefixes to reach the cluster, number of AS numbers that belong to cluster. Once hyper giants are determined, the dependency of popular websites on them will be examined by taking into consideration that the type of web objects like images, videos, HTML files etc delivered by these hyper giants. 

\subsubsection{Identifying hyper giants}

\subsubsection{case 1 :number of Links Vs number of IP addresses}
In this section we will take the top 3205 SLD infrastructures and cluster them based on their number of links to ip addresses they served.

\begin{figure}[h]
\includegraphics[width=\textwidth,height=10cm]{/home/sakib/soumya/wholeSLD/Rplot.png}
\centering
\caption{Clustering based on links and IP address features}
\end{figure}

We used k-means clustering algorithm and number of cluster parameter 10.We found 6 different clusters which are clubbed total 26 SLD infrastructures and showing unique behavior.Like cloudflare.net is clustered separately as it is serving very huge number of links as well as having very high number of ip addresses.It means lots of small SLDs are serving through cloudflare.net and it has footprint all over the world.Similarly us-east-1.elb.amazonaws.com clustered separately as it has less high of links but serving a high number of ip addresses.third cluster contain google.com which is serving high number of links but not very high number of ip addresses.In this way we are able to identify total 6 clusters which resulted in a total of 26 SLD infrastructures. But it is difficult to categorize them into some specific type of infrastructure based on only  links to ip address analysis. Hence these 26 SLD infrastructures will be further analyses taking into account their prefixes to their asn numbers.

\subsubsection{case 2 :number of BGP Prefixes Vs number of ASNs}
From last section we identify the SLD infrastructures which are having unique behavior. But we couldn't able to classify them .Therefore in this step we will check how they are distributed all other world.This requires these clusters to be analyzed using their corresponding ASN to prefix numbers.This is because number of ip prefixes shows the footprint of the infrastructures across the world and the number of asn numbers show the degree of distribution of infrastructures across the world.
\begin{figure}[h]
\includegraphics[width=\textwidth,height=10cm]{/home/sakib/soumya/wholeSLD/prefixVsasn.png}
\centering
\caption{Classification of hyper giants}
\end{figure}

From figure-14 ,we can cluster all the clustered SLD infrastructures into 5 parts based on their <number of prefixes,number of ASNs> analysis as below.

\begin{itemize}
\item very high,very high :
In total 3 different clustered SLD infrastructures are clustered under this.
They are yunjiasu-cdn.net,jiashule.com,
us-east-1.elb.amazonaws.com.These three SLD infrastructures contain very high number of prefixes as well as they have high number of ASN numbers,which shows that they have footprint all over the world  as well as they are distributed across the world.We can classify them as highly distributed CDNs.
\item high,high : 
Total 7 SLD infrastructures clubbed inside this. wpengine.com
,alikunlun.com,cloudflare.net,ourwebpic.com,
us-west-2.elb.amazonaws.com,eu-west-1.elb.amazonaws.com and 
ap-northeast-1.elb.amazonaws.com.They have high number of footprint and high number of asn numbers.This shows they have presence in few of the regions.We can classify them as distributed CDNs.
\item high,less  :
netdns-cdn.com and cdntip.com
These SLD infrastructures have high number of prefixes but have less number of ASN numbers.It means they have footprint all over the region but they normally administered through very few ASN numbers.Hence these can be categorized into cloud computing infrastructures. 
\item  medium,medium :
Five different SLD infrastructures are clubbed into same cluster.They are,
akamaiedge.net,ap-southeast-1.elb.amazonaws.com,
kxcdn.com,incapdns.net,d2t8dj4tr3q9od.cloudfront.net
 All these infrastructures have few prefixes and they also not distributed which gives an evidence of multi homes data centers or web hosting companies.
\item less,less :
The rest of the CDN infrastructures are clubbed into same cluster which having very few number of prefixes also very few number of IP addresses.Hence they can be treated as content providers.Google and Microsoft both clustered under this.
\end{itemize}

\pagebreak
\subsubsection{Conclusion}
From the above section,we are able to identify total 26 hyper giants which are having influence in Europe region.They are highly distributed CDNs,cloud platforms,CDNs,content producers etc.In next section we will see how they influence on number of objects delivered by them.

\subsection{Popular websites dependency on hyper giants}
In this section,first we will see different types of objects delivered through 219604 unique SLDs and compare this with web objects delivered by 26 hyper giants. Then we will examine each hyper giant separately and try to find what kind of data object they deliver. 
\subsubsection{Object types delivery through whole SLD infrastructures Vs hyper giants}
\begin{figure}[h]
\includegraphics[width=\textwidth,height=10cm]{/home/sakib/soumya/wholeSLD/objectCount.png}
\centering
\caption{top 10 object percentage used by all SLDs}
\end{figure}
\begin{figure}[h]
\includegraphics[width=\textwidth,height=10cm]{/home/sakib/soumya/wholeSLD/objectCount/hypergiant.png}
\centering
\caption{top 10 object percentage used by all hyper giants}
\end{figure}

Figure-15 and figure-16 shows top 10 object types in terms of their percentage of the total web objects they served .In case of hyper giants top 10 object types deliver almost 98.82 \% of the total object types they host. It can be inferred from the graph that hyper giants host 68.27\% of HTML object type making it the single highest contributor to the total web objects it hosts. 


Furthermore, 75.04\% of the total object types it hosts are of text object type such as text/html,text/css,text/xml,text/js etc. Furthermore, 18\% of the total object types it hosts are of image object type such as image/png and image/jpeg. 

, are delivered through hyper giants where as only 19.29\% of image files delivered through hyper giants.



In case of SLDs,top 10 object types carries almost 97.94\% of all object types  which is very much similar to the percentage of objects delivered by hyper giants.The highest web object type delivered through all SLDs as well as hyper giants is text/html but it can be observed that in case of all SLDs,text/html carries almost 38\% of all web objects which is around 68.27\% in case of delivering through hyper giants.In case of all SLDs ,object types are distributed properly.If we add image/jpeg,image/png,image/gif then all SLDs serve more image files than HTML files.But same is not the case for hyper giants.Similarly it can be seen that almost 46.77\% of text files delivered through whole sld set which is almost 1.5 times more in case of hyper giants.But in case of image files ,SLDs serve almost 42 \% of all web objects which is almost double the image files served through hyper giants.This different behavior might be because popular content websites normally store more dynamic files in CDNs where as in case of image files they store at their own servers.

\subsubsection{Object types delivered from hyper giants to popular web sites}
In this section we will see what kind of data mostly delivered through the identified hyper giants.In today's Internet ,content plays most vital role.Hence it is important to observe what kind of data mostly delivered through hyper giants .
\newline
\newline
\begin{tabular}{ |p{6cm}||p{2cm}|p{2cm}|p{2cm}| }
 \hline
 \multicolumn{4}{|c|}{Hyper giant Object List} \\
 \hline
 Country Name     & text&image&application\\
 \hline
 alikunlun.com   & 75.99    &20.32&   3.65 \\
d2t8dj4tr3q9od.cloudfront.net&   49.86  & 41.10   &8.68\\
 ap-northeast-1.elb.amazonaws.com &90.28 & 6.23&  3.44\\
 ap-southeast-1.elb.amazonaws.com    &87.94 & 7.36&  4.62\\
 cdntip.com&   87.94  & 7.36&1.67\\
 d5nxst8fruw4z.cloudfront.net& 31.90  & 57.42   &10.23\\
 eu-west-1.elb.amazonaws.com& 87.25  & 7.32&5.39\\
 fastlylb& 69.79  &21.30&8.69\\
 google.com&82.54  & 15.79&1.65\\
 jiashule.com& 88.85  & 8.33&2.80\\
 kxcdn.com& 75.62  & 17.98&6.34\\
pbwstatic.com& 83.18  &16.57&0.24\\ 
 akamaiedge.net&73.65  & 20.35&5.93\\ 
 anycast.me& 79.64  & 14.90&5.37\\
 cloudflare.net& 64.86  & 11.98&23.12\\
 cloudflare.com& 78.15  & 15.62&1.70\\ 
dynect.net& 94.53  & 3.74&3.74\\
 edgecastcdn.net& 53.23 & 38.44&8.17\\ 
incapdns.net& 64.86  & 7.61&4.81\\
 netdna-cdn.com& 21.08  & 55.04&23.39\\ 
ourwebpic.com& 86.56  & 11.58&1.84\\
 us-east-1.elb.amazonaws.com& 88.99  & 6.18&4.73\\ 
wpengine.com& 80.67  & 12.42&6.84\\
 us-west-2.elb.amazonaws.com&85.30  & 8.15&6.47\\ 
windows.net& 80.52  & 12.99&6.41\\
 yunjiasu-cdn.net& 87.76  & 9.86&0.0\\ 
 \hline
\end{tabular}
\newline

The table shows all 26 hyper giants and the percentage of text,image and application web objects delivered by them.We can see from table than most of the hyper giants deliver very high percentage of text files which contain text/html,text/css etc.But their are few exceptions .Like both the cloudfront SLDs are providing very high number of images compare to other hyper giants.  Similar kind of observation can be seen for netdna-cdn which provides more image files than text files.

From last section we observed yunjiasu-cdn.net,jiashule.com, us-east-1.elb.amazonaws.com
 are highly distributed CDNs.It can be observed that all the three CDNs are delivering very high number of text files compare to image and application files.This might be because of their footprint all over the world and also have massively distributed CDNs.Hence they cache more of the HTML  files at edge servers to provide better performance.Similarly observation can be seen from  distributed CDNs .These CDNs also deliver high number of HTML files compare to image files but as they have presence in some regions the number of HTML files are not that comparable to highly distributed CDNs.Third infrastructure type we observed was cloud computing infrastructures and netdns-cdn.com,cdntip.com clustered under that.From the table it can seen that  netdns-cdn.com delivers more images and very less number of HTML files.Again cdntip delivers highest number of application data.The data centers provide both images and HTML files in a  very balance way.cloud front delivers 49\% of HTML file and 41\% of image files.Similarly akamaiedge.net provide around 20\%  of images.Content providers like window.net and google.com delivers very high number of links compare to number of images.This is evident as they have more content . 

\subsubsection{Conclusion}
From the section we can infer that both SLDs and hyper giants deliver maximum number of text/HTML files but it also found that hyper giants delivered almost 1.5 times more HTML files than SLDs.Same kind of observation can be seen for image files where SLDs deliver double the image content than hyper giants.Again we found that highly distributed CDN generally deliver more HTML links compare to image or application files which might be because of their massive CDN distribution.Distributed CDNs provide high number of HTML files because of their  presence in few regions.Cloud computing cdns provide more images than HTML file which might be because cloud computing provide scalability.Data centers delivers both HTML file and images in almost same ratio.Content providers also delivers more HTML files compare to image files which because of their rich content.
