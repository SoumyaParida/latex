In this section,the prominent hosting infrastructures will be identified first.Next these prominent infrastructures will be clustered together using the cluster algorithm  described in chapter-3.Once clustered SLD infrastructures are identified,those will be analyzed further to gain insight on the their deployment and hosting strategies.Finally based on their strategies hyper giants will be determined.Once hyper giants will be determined ,the dependency between them and popular websites will be examined by taking into consideration that what type of web objects like images,videos,HTML files etc delivered through these hyper giants to popular websites. 

\subsection{Analyzing prominent infrastructures}
In this section we will analyses the SLD infrastructures present in Internet based on number of URLs served by them.A small subset of the SLDs are analyzed to understand if there are common characteristics present within them.
\begin{figure}[h]
\includegraphics[width=\textwidth,height=10cm]
{/home/sakib/soumya/wholeSLD/graphs/top20SLD}
\centering
\caption{Top 20 SLDs}
\end{figure}

From figure-11 ,it is found that there are some SLDs which are served by same company.Like akamaiedge.net and akamai.net both are CNAMEs used by Akamai company.Similarly google.com,googleusercontent.com,googlehosted.com and googledomains.com all are used by Google.Normally companies used different names when they serve different services to the customer or different name for certain  located customers.This can be seen from the CNAMEs used by amazonaws.
us-east-1.elb.amazonaws.com,eu-west-1.elb.amazonaws.com are two examples of naming pattern used by the amazon to provide services to distinct located customers.But here the question arises that these names are pointing to same infrastructure or different.If they are pointing to same infrastructure they should be considered as one ,else separately.Hence it is important to identify if these SLDs share same the infrastructure or not.This can be identified by analyzing the bgp prefixes they share.To measure this we are going to take RIPE bgp prefixes of each SLD routed to and cluster all the SLDs that use the same bgp prefixes.

The top 20 SLDs serve almost 13.13\% of all the URLs crawled.These 20 SLDs contain not only CDNs like Akamai, Cloudflare etc., but also contain content providers like Google,Facebook.It also contain SLDs like amazonaws which provides cloud computing services while some other SLDs are web hosting companies like ccgslb.net.A small fraction of data set gives us multiple types of infrastructures.Classification of different infrastructures cannot be done using only number of links.To know if a infrastructure is highly distributed all across the world,the number of ASN number need to be checked.

\subsubsection{Conclusion}
From this section the following observations can be inferred.
\begin{enumerate}
\item There are some SLDs which are served by same parent company like akamaiedge.net and akamai.net which are served by company Akamai.Hence it is important to identify whether they can be clustered into same infrastructure or not.This cannot be analyzed with just the number of URLs instead we need to check the footprint covered by these SLDs in the world by checking their bgp prefix routes. 
\item Since it is identified that there is a possibility of some SLDs getting clustered , it is important not to restrict the test sample for the top 20 of SLDs,but should be extended to all the 219604 SLDs.
\end{enumerate}
In the next section,we will discuss the steps to identify the hyper giants using the clustering algorithm, as discussed in chapter 3 methodology section.
\subsection{Identifying hyper giants}
In 2010, Craig Labovitz, then of Arbor Networks,characterized the hyper giant as a content provider that makes massive investments in bandwidth, storage, and computing capacity to maximize efficiencies and performance.But as the architecture of Internet evolves,researchers found that the Internet has now become a flatter infrastructure where there are fewer autonomous systems connected to each other and they try to have a bigger footprint in terms of number of bgp prefixes than before.In this way they are able to diversify their architecture as well as able to move content to even closer to their customers.They termed this infrastructure providers as hyper giants [14].

Overall we get total 219604 unique second level domains.But from them a lot of SLDs which can be clustered into other SLDs.

\begin{figure}[h]
\includegraphics[width=\textwidth,height=10cm]{/home/sakib/soumya/wholeSLD/graphs/clubbedSLDWhole.png}
\centering
\caption{Number of SLDs served by different SLD infrastructure clusters.}
\end{figure}

After clustering algorithm we are able to get 53852 unique clustered SLD infrastructures.From there ,almost 80\% of the total unclustered SLDs got clustered into first 3.73\% of SLDs .It shows that these 3.73 \% of top SLDs have footprint all over the world through highly distributed CDNs,data centers etc.Other SLDs share their infrastructures with these top 3.73\% of clustered infrastructures.Hence there is a possibility to get the hyper giants in these 3.73\% of clustered SLD infrastructures.But there are SLD infrastructures who have their own infrastructures in the form of data centers.Hence they don't share any other SLD infrastructures.Like facebook.com who has its own infrastructures in the form of data centers all over the world.In fact we also found almost 82.45\% of clustered infrastructures who does not share their infrastructure with no more than another SLD infrastructure.It means there are companies who work independently by creating their own infrastructures.This can be data centers all over world.Although these 82.45\% of clustered infrastructure do not share their infrastructure with no more than single infrastructure, still some of them serve a large number of links which make them another candidate for hyper giant analysis.

Hence to get a better picture we will see clustered SLD infrastructures based on how many links they served.So we sorted all clustered SLD infrastructures in their decreasing order of links they serve and we found that top 5.95 \% (=3205) clustered SLD infrastructures serve almost 80\% of links and have 78.65 \% of SLDs.Hence there is a possibility of getting hyper giants in this range.

\subsubsection{Clustered SLDs}
From last section we identified 5.95\% (=3205) clustered SLD infrastructures as candidates for hyper giant analysis.To identify the hyper giants,two different steps will be followed.In the first step,the 5.95\% clustered SLD infrastructures will be analyzed based on number of links they serve, to number of IP addresses they resolve.The big SLD cluster infrastructures will be separated from small SLD infrastructures by end of this step.In the next step we will again compare number of prefixes they resolved as a clustered SLD infrastructure to number of ASN numbers they belong to.This will give us a better idea how their whole infrastructures are distributed all over the world.After these two process we will try to identify the hyper giants.

\subsubsection{case 1 :number of Links Vs number of IP addresses}
In this section we will take the top 3205 SLD infrastructures and cluster them based on their number of links to ip addresses they served.

\begin{figure}[h]
\includegraphics[width=\textwidth,height=10cm]{/home/sakib/soumya/wholeSLD/Rplot.png}
\centering
\caption{Clustering based on links and IP address features}
\end{figure}

We used k-means clustering algorithm and number of cluster parameter 10.We found 6 different clusters which are clubbed total 26 SLD infrastructures and showing unique behavior.Like cloudflare.net is clustered separately as it is serving very huge number of links as well as having very high number of ip addresses.It means lots of small SLDs are serving through cloudflare.net and it has footprint all over the world.Similarly us-east-1.elb.amazonaws.com clustered separately as it has less high of links but serving a high number of ip addresses.third cluster contain google.com which is serving high number of links but not very high number of ip addresses.In this way we are able to identify total 6 clusters which resulted in a total of 26 SLD infrastructures. But it is difficult to categorize them into some specific type of infrastructure based on only  links to ip address analysis. Hence these 26 SLD infrastructures will be further analyses taking into account their prefixes to their asn numbers.

\subsubsection{case 2 :number of BGP Prefixes Vs number of ASNs}
From last section we identify the SLD infrastructures which are having unique behavior. But we couldn't able to classify them .Therefore in this step we will check how they are distributed all other world.This requires these clusters to be analyzed using their corresponding ASN to prefix numbers.This is because number of ip prefixes shows the footprint of the infrastructures across the world and the number of asn numbers show the degree of distribution of infrastructures across the world.
\begin{figure}[h]
\includegraphics[width=\textwidth,height=10cm]{/home/sakib/soumya/wholeSLD/prefixVsasn.png}
\centering
\caption{Classification of hyper giants}
\end{figure}

From figure-14 ,we can cluster all the clustered SLD infrastructures into 5 parts based on their <number of prefixes,number of ASNs> analysis as below.

\begin{itemize}
\item very high,very high :
In total 3 different clustered SLD infrastructures are clustered under this.
They are yunjiasu-cdn.net,jiashule.com,
us-east-1.elb.amazonaws.com.These three SLD infrastructures contain very high number of prefixes as well as they have high number of ASN numbers,which shows that they have footprint all over the world  as well as they are distributed across the world.We can classify them as highly distributed CDNs.
\item high,high : 
Total 7 SLD infrastructures clubbed inside this. wpengine.com
,alikunlun.com,cloudflare.net,ourwebpic.com,
us-west-2.elb.amazonaws.com,eu-west-1.elb.amazonaws.com and 
ap-northeast-1.elb.amazonaws.com.They have high number of footprint and high number of asn numbers.This shows they have presence in few of the regions.We can classify them as distributed CDNs.
\item high,less  :
netdns-cdn.com and cdntip.com
These SLD infrastructures have high number of prefixes but have less number of ASN numbers.It means they have footprint all over the region but they normally administered through very few ASN numbers.Hence these can be categorized into cloud computing infrastructures. 
\item  medium,medium :
Five different SLD infrastructures are clubbed into same cluster.They are,
akamaiedge.net,ap-southeast-1.elb.amazonaws.com,
kxcdn.com,incapdns.net,d2t8dj4tr3q9od.cloudfront.net
 All these infrastructures have few prefixes and they also not distributed which gives an evidence of multi homes data centers or web hosting companies.
\item less,less :
The rest of the CDN infrastructures are clubbed into same cluster which having very few number of prefixes also very few number of IP addresses.Hence they can be treated as content providers.Google and Microsoft both clustered under this.
\end{itemize}

\pagebreak
\subsubsection{Conclusion}
From the above section,we are able to identify total 26 hyper giants which are having influence in Europe region.They are highly distributed CDNs,cloud platforms,CDNs,content producers etc.In next section we will see how they influence on number of objects delivered by them.

\subsection{Popular websites dependency on hyper giants}
In this section,first we will see different types of objects delivered through 219604 unique SLDs and compare this with web objects delivered by 26 hyper giants. Then we will examine each hyper giant separately and try to find what kind of data object they deliver. 
\subsubsection{Object types delivery through whole SLD infrastructures Vs hyper giants}
\begin{figure}[h]
\includegraphics[width=\textwidth,height=10cm]{/home/sakib/soumya/wholeSLD/objectCount.png}
\centering
\caption{top 10 object percentage used by all SLDs}
\end{figure}
\begin{figure}[h]
\includegraphics[width=\textwidth,height=10cm]{/home/sakib/soumya/wholeSLD/objectCount/hypergiant.png}
\centering
\caption{top 10 object percentage used by all hyper giants}
\end{figure}

Figure-15 and figure-16 shows top 10 object types in form of percentage delivered by both hype giants and slds.In case of hyper giants top 10 object types deliver almost 98.82 \% of all object types.Again we can see that most of these object types are HTML files.HTML carries almost 68.27\% of object type in compare to other object types.Similarly 75.04\% of  text object types which contain text/html,text/css,text/xml,text/js etc., are delivered through hyper giants where as only 19.29\% of image files delivered through hyper giants.

In case of SLDs,top 10 object types carries almost 97.94\% of all object types  which is very much similar to the percentage of objects delivered by hyper giants.The highest web object type delivered through all SLDs as well as hyper giants is text/html but it can be observed that in case of all SLDs,text/html carries almost 38\% of all web objects which is around 68.27\% in case of delivering through hyper giants.In case of all SLDs ,object types are distributed properly.If we add image/jpeg,image/png,image/gif then all SLDs serve more image files than HTML files.But same is not the case for hyper giants.Similarly it can be seen that almost 46.77\% of text files delivered through whole sld set which is almost 1.5 times more in case of hyper giants.But in case of image files ,SLDs serve almost 42 \% of all web objects which is almost double the image files served through hyper giants.This different behavior might be because popular content websites normally store more dynamic files in CDNs where as in case of image files they store at their own servers.

\subsubsection{Object types delivered from hyper giants to popular web sites}
In this section we will see what kind of data mostly delivered through the identified hyper giants.In today's Internet ,content plays most vital role.Hence it is important to observe what kind of data mostly delivered through hyper giants .
\newline
\newline
\begin{tabular}{ |p{6cm}||p{2cm}|p{2cm}|p{2cm}| }
 \hline
 \multicolumn{4}{|c|}{Hyper giant Object List} \\
 \hline
 Country Name     & text&image&application\\
 \hline
 alikunlun.com   & 75.99    &20.32&   3.65 \\
d2t8dj4tr3q9od.cloudfront.net&   49.86  & 41.10   &8.68\\
 ap-northeast-1.elb.amazonaws.com &90.28 & 6.23&  3.44\\
 ap-southeast-1.elb.amazonaws.com    &87.94 & 7.36&  4.62\\
 cdntip.com&   87.94  & 7.36&1.67\\
 d5nxst8fruw4z.cloudfront.net& 31.90  & 57.42   &10.23\\
 eu-west-1.elb.amazonaws.com& 87.25  & 7.32&5.39\\
 fastlylb& 69.79  &21.30&8.69\\
 google.com&82.54  & 15.79&1.65\\
 jiashule.com& 88.85  & 8.33&2.80\\
 kxcdn.com& 75.62  & 17.98&6.34\\
pbwstatic.com& 83.18  &16.57&0.24\\ 
 akamaiedge.net&73.65  & 20.35&5.93\\ 
 anycast.me& 79.64  & 14.90&5.37\\
 cloudflare.net& 64.86  & 11.98&23.12\\
 cloudflare.com& 78.15  & 15.62&1.70\\ 
dynect.net& 94.53  & 3.74&3.74\\
 edgecastcdn.net& 53.23 & 38.44&8.17\\ 
incapdns.net& 64.86  & 7.61&4.81\\
 netdna-cdn.com& 21.08  & 55.04&23.39\\ 
ourwebpic.com& 86.56  & 11.58&1.84\\
 us-east-1.elb.amazonaws.com& 88.99  & 6.18&4.73\\ 
wpengine.com& 80.67  & 12.42&6.84\\
 us-west-2.elb.amazonaws.com&85.30  & 8.15&6.47\\ 
windows.net& 80.52  & 12.99&6.41\\
 yunjiasu-cdn.net& 87.76  & 9.86&0.0\\ 
 \hline
\end{tabular}
\newline

The table shows all 26 hyper giants and the percentage of text,image and application web objects delivered by them.We can see from table than most of the hyper giants deliver very high percentage of text files which contain text/html,text/css etc.But their are few exceptions .Like both the cloudfront SLDs are providing very high number of images compare to other hyper giants.  Similar kind of observation can be seen for netdna-cdn which provides more image files than text files.

From last section we observed yunjiasu-cdn.net,jiashule.com, us-east-1.elb.amazonaws.com
 are highly distributed CDNs.It can be observed that all the three CDNs are delivering very high number of text files compare to image and application files.This might be because of their footprint all over the world and also have massively distributed CDNs.Hence they cache more of the HTML  files at edge servers to provide better performance.Similarly observation can be seen from  distributed CDNs .These CDNs also deliver high number of HTML files compare to image files but as they have presence in some regions the number of HTML files are not that comparable to highly distributed CDNs.Third infrastructure type we observed was cloud computing infrastructures and netdns-cdn.com,cdntip.com clustered under that.From the table it can seen that  netdns-cdn.com delivers more images and very less number of HTML files.Again cdntip delivers highest number of application data.The data centers provide both images and HTML files in a  very balance way.cloud front delivers 49\% of HTML file and 41\% of image files.Similarly akamaiedge.net provide around 20\%  of images.Content providers like window.net and google.com delivers very high number of links compare to number of images.This is evident as they have more content . 

\subsubsection{Conclusion}
From the section we can infer that both SLDs and hyper giants deliver maximum number of text/HTML files but it also found that hyper giants delivered almost 1.5 times more HTML files than SLDs.Same kind of observation can be seen for image files where SLDs deliver double the image content than hyper giants.Again we found that highly distributed CDN generally deliver more HTML links compare to image or application files which might be because of their massive CDN distribution.Distributed CDNs provide high number of HTML files because of their  presence in few regions.Cloud computing cdns provide more images than HTML file which might be because cloud computing provide scalability.Data centers delivers both HTML file and images in almost same ratio.Content providers also delivers more HTML files compare to image files which because of their rich content.
