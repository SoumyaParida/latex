\section{Results\label{cha:chapter6}}
\noindent This chapter analyzes the measurements obtained from chapter \ref{cha:chapter5} to identify hyper giants by studying the features of clustered hosting infrastructure such as number of links, number of IP addresses,  number of BGP prefixes, number of AS numbers that belong to cluster. Once hyper giants are determined, the dependency of popular websites on them will be examined by analyzing the type of web objects like images, videos, HTML files etc. they deliver.\\

\subsection{Identifying hyper giants}
\noindent This section analyses the 3205 clustered hosting infrastructures using their feature set to identify if there are any hyper giants. The hosting infrastructure that has unique features can be used to classify them as  hyper giants. Hence some mining technique has to be applied to separate the hyper giants from the rest of the hosting infrastructure. For this purpose, k-means clustering is used to separate infrastructure into different clusters each having it's own characteristics. Furthermore, these clusters are analyzed to identify if any of their characteristics can qualify them to be treated as hyper giants.\\ 

\noindent K-means algorithm uses, the number of links hosted by each hosting infrastructure and the number of IP addresses each hosting infrastructure contain as it's inputs. The k-means algorithm clusters the given 3205 hosting infrastructure. The value of of k=10 is chosen as a clustering parameter as it gave clusters with unique characteristics. The following graph as shown in Figure \ref{fig:cluster} represents the distribution of hosting infrastructure into clusters, which is obtained using k-means algorithm.\\

\begin{figure}[htb]
  \centering
  \includegraphics[width=14cm]{/home/sakib/soumya/wholeSLD/Rplot.png}\\
  \caption{Distribution of hosting infrastructure into clusters}
  \label{fig:cluster}
\end{figure}

\noindent The details of the 10 clusters mined from the 3205 hosting infrastructure that we considered for analysis are categorized into 10 clusters using k-means, whose information is tabulated in Table \ref{tab:top10}.\\\

\begin{table}[htb]
\centering
\begin{tabular}[t]{|p{.1\textwidth}|p{.2\textwidth}||p{.3\textwidth}|}
\hline
 cluster No.& No. of hosting infrastructure&Link density of cluster\\
 \hline
 1&1& 1295505\\
 2&3& 149042\\
 3&3015&1386\\
 4&1&597533\\
  5&8&72584\\
   6&1& 240910\\
   7&130&13642\\
    8&34&16735\\
     9&1&199109\\
      10&11&40648\\
 \hline
\end{tabular}
\caption{k-means clustering}
\label{tab:top10}
\end{table}

\noindent As observed from Table \ref{tab:top10}, the number of hosting infrastructures inside clusters 1, 2, 4, 5, 6, 9 and 10 are very sparsely distributed, while in clusters 3,7 and 8, the hosting infrastructure is very densely distributed. Thus the clusters 1, 2, 4, 5, 6, 9, 10  can be treated separately from the other clusters owing to the fact that their hosting infrastructure distribution in these two groups are completely different.\\

\begin{figure}[htb]
  \centering
  \includegraphics[width=14cm]{/home/sakib/soumya/wholeSLD/geo/linkdensity.png}\\
  \caption{link density per cluster}
  \label{fig:hosting}
\end{figure}

\begin{table}[htb]
\centering
\begin{tabular}[t]{|p{.1\textwidth}||p{.6\textwidth}|}
\hline
 sl. no. &Hyper giants\\
 \hline
 1&alikunlun.com \\
2&d2t8dj4tr3q9od.cloudfront.net\\
 3&ap-northeast-1.elb.amazonaws.com\\
 4&ap-southeast-1.elb.amazonaws.com\\
 5&cdntip.com\\
 6&d5nxst8fruw4z.cloudfront.net\\
 7&eu-west-1.elb.amazonaws.com\\
 8&fastlylb\\
 9&google.com\\
 10&jiashule.com\\
 11&kxcdn.com\\
12&pbwstatic.com\\ 
 13&akamaiedge.net\\ 
 14&anycast.me\\
 15&cloudflare.net\\
 16&cloudflare.com\\ 
17&dynect.net\\
 18&edgecastcdn.net\\ 
19&incapdns.net\\
 20&netdna-cdn.com\\ 
21&ourwebpic.com\\
 22&us-east-1.elb.amazonaws.com\\ 
23&wpengine.com\\
 24&us-west-2.elb.amazonaws.com\\ 
25&windows.net\\
 26&yunjiasu-cdn.net\\ 
 \hline
\end{tabular}
\caption{26 hyper giants}
\label{tab:hypergiant}
\end{table}
\pagebreak

\noindent Link density of a cluster can be defined as the average no of links of the cluster, which is obtained by diving the number of links hosted by cluster with the total number of hosting infrastructure in the cluster. Figure \ref{fig:hosting} shows the distribution of Link density for each cluster. X-axis of the graph shown in Figure 10 represents the cluster number, while the Y-axis represents the Link density of the cluster. \\

\noindent From the graph show in Figure \ref{fig:hosting}, It is clear that clusters 1, 2, 4, 5, 6, 9,10 have relatively higher link densities compared to the other clusters. Hence these clusters can be grouped together, which here afterwards referred to as group-1, while the rest of the clusters 3,7,8 can be grouped together, and be called here afterwards as group-2.\\

\noindent Since, we regrouped the total 10 clusters in 2 groups, it has to be analyzed further to identify which one of these two groups can be qualified as a hyper giant. This can evaluated by analyzing the link density of each group. Link density of a group can be calculated by summing the total number of links hosted by each cluster of the group and dividing this sum with total number of hosting infrastructure of the group. In this case, group-1 has 7 clusters, who has a cumulative total of  4088821 links and 26 hosting infrastructure. While in group-2, which has 3 clusters, has a cumulative total of  6959026 links and 3179 hosting infrastructure. Hence, the link density of the group-1 is 157262.36 (4088821/26), and for group-2 is 2189.06 (6959026/3179). From this, it is observed that their link densities are approximately in the ratio of 72:1, thus qualifying the members of group-1 to be called as hyper giants.\\

\noindent 26 hosting infrastructure are concluded as hyper giants shown in Table \ref{tab:hypergiant}, which are collectively hosting 33,73\% of the total web links crawled.\\

\subsubsection{Conclusion}
\noindent From the above section,we are able to identify total 26 hyper giants when analyzed from a single vantage point in Germany. These hyper giants are collectively hosting 33,73\% of the total number of links crawled. As the study has considered 100,000 top ranked websites, it can concluded that these hyper giants are hosting approximately 33,73\% of Internet web traffic. \\

\subsection{Popular websites dependency on hyper giants}
\noindent Hyper giants build a large infrastructure all around the world to deliver content ensuring a faster response. The websites use these infrastructure to store their content, such as audios ,videos, test/html files etc. Therefore, any kind of disruption in serving these contents from hyper giants will impact the websites which shows the interdependency of hyper giants with the web sites.
This section focuses on finding how hyper giants influence the delivery of web objects of Internet web traffic. \\
 
\subsubsection{Cumulative object type distribution in hyper giants}
\noindent The goal of this study is to analyze how hyper giants influence the delivery of web objects. Figure \ref{fig:Topobject} depicts the comparison between the number of web objects delivered by all the hosting infrastructure put together against the ones by hyper giants alone. The x-axis in the graph shows the type of web object delivered and the y-axis represents the number of web objects delivered. For better view, all the web object types are categorized into 4 major types. They are \enquote{text}, \enquote{image}, \enquote{application}, \enquote{other} types. The \enquote{text} object type contains different object types such as text/html, text/css etc. The \enquote{image} object contain objects such as image/png, image/jpeg etc. Similarly the \enquote{application} type object contains objects such as application/javascript, application/asp etc.

\begin{figure}[htb]
  \centering
  \includegraphics[width=14cm]{/home/sakib/soumya/testObject/objectCompare.png}\\
  \caption{top 10 object percentage used by all hyper giants}
  \label{fig:Topobject}
\end{figure}

\noindent From the Figure \ref{fig:Topobject}, it can be observed that, out of total 12,120,731 links crawled from 100,000 top ranked web sites of Alexa, 9,750,438 are of \enquote{text} object type. 1,741,064 are of \enquote{image} type, 620,573 are of \enquote{application} type and 8656 are of \enquote{other} type while considering all hosting infrastructure. Similarly out of total 9,750,438 text objects, hyper giants are delivering 3,183,272 text objects which is 32,64 \%. Out of the total 1,741,064 image objects, 682,962 image objects delivered by hyper giants which is 39,22 \% of all image objects. For out of 620,573 application type application objects, hyper giants are delivering 219,801 application objects which is 35,41 \% of all application objects. Similarly for the category type \enquote{other}, hyper giants are delivering 2786 out of the total 8656 objects delivered through hosting infrastructure put together. When all the object types are put together, hyper giants host a total of  4,088,821 web objects, which is 33,73 \% of all the web objects hosted by all the infrastructure put together.

\subsubsection{Individual object type distribution in hyper giants }
\noindent In this section, different types of web objects delivered through the identified hyper giants are analyzed. In today's Internet ,content plays a vital role. Hence it is important to analyze what kind of data is delivered by each hyper giant.\\

\begin{table}[htb]
\centering
\begin{tabular}[t]{|p{.4\textwidth}|p{.1\textwidth}||p{.1\textwidth}||p{.2\textwidth}||p{.1\textwidth}|}
\hline
 Country Name     & text&image&application&other\\
 \hline
dynect.net& 94.71& 3.76& 1.52& 0.01\\
ap-northeast-1.elb.amazonaws.com& 90.09& 6.44& 3.43& 0.03\\
cdntip.com& 89.55& 8.79& 1.66& 0.0\\
yunjiasu-cdn.net& 89.07& 8.56& 2.36& 0.01\\
jiashule.com& 88.89& 8.73& 2.38& 0.0\\
eu-west-1.elb.amazonaws.com& 88.29& 6.76& 4.93& 0.02\\
us-east-1.elb.amazonaws.com& 87.73& 7.19& 5.0& 0.08\\
incapdns.net& 87.39& 7.74& 4.85& 0.02\\
ap-southeast-1.elb.amazonaws.com& 86.74& 8.72& 4.43& 0.11\\
ourwebpic.com& 85.37& 12.65& 1.98& 0.01\\
pbwstatic.com& 84.46& 15.29& 0.25& 0.0\\
us-west-2.elb.amazonaws.com& 84.21& 8.68& 7.04& 0.07\\
wpengine.com& 81.15& 12.01& 6.79& 0.05\\
windows.net& 80.1& 13.23& 6.39& 0.28\\
anycast.me& 78.82& 15.91& 5.19& 0.08\\
cloudflare.net& 78.63& 15.26& 6.05& 0.06\\
alikunlun.com& 78.37& 18.54& 3.07& 0.02\\
cloudflare.com& 76.34& 15.24& 8.39& 0.03\\
kxcdn.com& 75.79& 17.9& 6.27& 0.05\\
akamaiedge.net& 75.5& 18.63& 5.8& 0.07\\
google.com& 75.07& 23.2& 1.72& 0.01\\
fastlylb.net& 69.02& 22.43& 8.4& 0.14\\
d2t8dj4tr3q9od.cloudfront.net& 50.26& 41.64& 7.77& 0.33\\
d5nxst8fruw4z.cloudfront.net& 32.83& 57.3& 9.43& 0.44\\
netdna-cdn.com& 19.44& 59.04& 21.14& 0.37\\
 \hline
\end{tabular}
\caption{web object distribution for hyper giants}
\label{tab:hyperObject}
\end{table}

\noindent Table \ref{tab:hyperObject} shows the list of all the 26 hyper giants and their corresponding object distribution in terms of the percentage of the total web objects they serve. It can be observed from table that most of the hyper giants deliver very high percentage of text files which contain text/html,text/css etc. But there are few exceptions, such as the hosting infrastructure \enquote{netdna-cdn.com}, \enquote{d5nxst8fruw4z.cloudfront.net} and \enquote{netdna-cdn.com} are providing very high number of images compare to other hyper giants.In comparison to other hyper giants, it can be observed that \enquote{netdna-cdn.com} host highest number of "application" type of objects.\\

 \subsubsection{Conclusion}
\noindent From the section, it can be inferred that the percentage of \enquote{text}, \enquote{image}, \enquote{application}, \enquote{other} are almost percentage wise align with the overall percentage of web object delivered through hyper giant compare to web object delivered through all the hosting infrastructure which is 33,73 \%. For object distribution, most of the hyper giants deliver more text files compare to other types of objects, except for \enquote{d5nxst8fruw4z.cloudfront.net} and \enquote{netdna-cdn.com} who delivers more image type of objects.
\clearpage