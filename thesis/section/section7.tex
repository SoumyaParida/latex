In this thesis,we introduce a automated process to find out the prominent infrastructures in today's Internet as well as to classify these prominent infrastructures to find out the presence of hyper-giants using DNS measurement and bgp prefixes.We presented a clustering algorithm which will help to find out which SLDs are sharing their infrastructures.The advantage of this automated approach is that it uses each to retrieve SLD and bgp prefixes ,hence this procedure can be used in future.

Along with this we measure what object types are delivered by major hyper giants.This will help researchers to get a better view to classify hyper giants based on their object type delivery.Not all popular websites provide same kind of content.Some websites are popular for delivering videos and some other are for user content.Hence with this change of content type,we provide a overview of hyper giants according to different object type they serve.

The data is collected at a single vantage point at Germany.This thesis was able to identify high distributed CDNs,cloud service providers,content providers etc. and their role which will mostly hold good for across Europe.

Furthermore our thesis is an important step towards answering some of the very crucial questions for highly distributed CDNs,distributed CDNs,content providers etc.It will give them idea to find out how other CDNs distribute their infrastructure as well as their network footprint distribute across different region which will give them a competitive advantage over their competitors in content delivery market place.Moreover it will help the research community to discover the Internet architecture changes with time.They also can able to track the hyper giants and their dependency with other popular websites.
