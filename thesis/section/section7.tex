\section{Conclusion\label{cha:chapter7}}
\noindent This thesis conducted an empirical analysis to find out the presence of hyper giants on Germany's Internet ecosystem. It aims to answer the following research questions: Firstly, whether there is any presence of hyper giants in today's Internet? Secondly, what web objects contained in popular websites are delivered through hyper giants? \\

\noindent To answer the first question, we introduced an automated process to find out the hosting infrastructure in today's Internet as well as cluster these hosting infrastructure to find out the presence of hyper giants. To perform this task, 100,000 top ranked websites of Alexa are crawled using scrapy engine and their HTML codes are inspected to retrieve all their embedded URLs. Using DNS resolution and HTTP header analysis, different features of hosting infrastructure are collected such as number of links, number of IP addresses, number of BGP prefixes, number of ASNs. A clustering algorithm is presented which will help to find out which hosting infrastructure are sharing infrastructure with each other using features of hosting infrastructure. The advantage of this automated approach is that, it uses SLD and BGP prefixes to perform the tasks, hence this procedure can be used in any other research. The study reveals that there are 26 hyper giants present in Internet when analyzed from a single vantage point in Germany, which are collectively contributing to nearly 30\% percentage of total web traffic on Internet. Some well known hosting infrastructure are detected in these 26 hyper giants, such as Google, Akamai, Amazon, Cloudflare etc.\\

\noindent The thesis has also focused on finding out, what web objects contained in popular websites are delivered through hyper giants. Using HTTP header analysis, different web object types such as text/html, image, video, audio etc. are collected. It is found that majority of the hyper giants focuses on delivering more number of text/HTML object files than other object types. This reveals that any disruption on hyper giant infrastructure may leave many web pages being unloaded. It is found that cloudflare.net delivered maximum percentage of \enquote{application} type web objects compared to other type of objects that it hosts. Google, Akamai are hosting more text/html object type files compared to other object types. cloudfront.com and netdna-cdn.com are delivering more image files compare to other web object types they host. This information will help the websites, which are focused on delivering a particular kind of content. For example video sharing sites are mainly focused on video content where as blogs focus more on text/html objects.This information might be of some value for this kind of web sites while they are making a decision on where to host their content. \\

\noindent The data is collected at a single vantage point in Germany. Furthermore this thesis is an important step towards answering some of the very crucial questions for CDNs, content providers etc. Moreover it will help the research community to discover the Internet architecture changes with time. They also can able to track the hyper giants and dependency of popular websites in these hyper giants.\\

The following areas can be investigated further which are not covered in the current scope of this thesis.

\begin{itemize}
\item Measurements are performed on a single vantage point in Germany. Hence the identification of hyper giants, their role and the degree of dependency of popular websites on hyper giants might change when the whole procedure will be carried out from multiple vantage points across the world. 

\item Secondly while clustering the hyper giants,the k-means parameter is taken by going through very small number of observations.Hence in future this can be tested more precisely which might help to cluster the hosting infrastructures at a granular level.

\item In the clustering algorithm, a similarity index of 0.7 is chosen to decide that two infrastructure can be clustered together. While this is an assumption, further analysis can be done on choosing the right similarity index.

\end{itemize}
