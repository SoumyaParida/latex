\section{Background\label{cha:chapter2}}

\noindent This chapter gives an overview of the Internet architecture, describes the functionality of each of its elements, with the focus on the role of hyper giants in today's Internet. Different studies related to hyper giants are discussed. Finally, the technical background to Domain name system (DNS) and hypertext transfer protocol (HTTP) that are needed to carried out the required  study in this thesis are explained briefly.\\

\subsection{Related studies\label{sec:related}}
\noindent The Internet architecture is getting complex with time with introduction of hyper giants. Therefore it is very important to identify them and study their role in Internet. In 2010, Craig Labovitz, then of Arbor Networks \cite{Labovitz}, first time characterized hyper giant. By placing Google in this list, the author described the hyper giant as, a content provider that makes massive investments in bandwidth, storage, and computing capacity to maximize efficiencies and performance. It is also found that the traffic amount sent by hyper giants is about 30\% of the whole amount across the Internet. This concept of hyper giants also aligns with Schmidts \cite{Schonfeld} assertion which talks about ”gang of four” companies which are responsible for the growth and innovation of Internet. Google, Apple, Amazon, and Facebook. In \cite{Shavitt}, Shavitt and Weinsberg analyzed changes in topological structure, such as betweenness centrality and link density, by focusing on large content providers, also referred to as hyper giants \cite{Labovitz} \cite{Poese}. They create a snapshot of the AS-level graph from late 2006 until early 2011, and then analyze the interconnection trends of the transit and content providers and their implications for the Internet ecosystem. Shavitt and Weinsberg proved that large content providers like Google, Yahoo!, Microsoft, Facebook, and Amazon have increased their connectivity degree during the observed period and are becoming key players in the Internet ecosystem, strengthening the idea that the Internet is becoming flatter. From this analysis, it was found that the structure of the Internet topology has changed from a hierarchical to a flat structure. This is because large content providers construct links with a lot of small ISPs. Unfortunately all the above studies are carried out by accessing the Internet traffic passing through different ASs, not from end user.

\subsection{Evolution of Internet Architecture and Rise of hyper giants\label{sec:archi}}
\noindent The Internet architecture implemented until the early 2000s was based on a multi-tier hierarchical structure. Tier-1 ISPs were on top of the hierarchy followed by the Tier-2 regional ISPs and the Access ISPs at the lower part of the hierarchy connecting the end users.Tier-1 were highly connected to other ISPs and offered transit services to other ISPs in lower layers. Content was distributed through Access ISPs or, in the best cases, through ISPs located at advantageous points. Traffic flows were required to go up and then down in the hierarchy to reach end users shown in Figure \ref{fig:interArchiOld}. Among different network operators, Internet traffic was exchanged at different IP exchange points according to agreements between different layer players where the dissymmetry in traffic was compensated \cite{Krishnan1} \cite{Bernhard} \cite{Ager}. \\

\begin{figure}[htb]
  \centering
  \includegraphics[width=14cm]{/home/sakib/soumya/latexNew/latex/internetArchiOld.png}\\
  \caption{Traditional Hierarchic Internet Structure}
  \label{fig:interArchiOld}
\end{figure}

\noindent It can be observed from Figure \ref{fig:interArchiOld}, Tier-1 operators are at top of this hierarchy. Tier-1 network is normally a transit free network that peers with every other Tier-1 network operators. Hence they treat each other equally. While most of the Tier-1 operators offers global coverage, there are some which restricted geographically. Tier-2 ISP is a customer of Tier-1 operators and pays to Tier-1 operators for connectivity to rest of the Internet. They connect to one or more Tier-1 operators, possibly to other Tier-2 operators. Below to the Tier-2 ISPs are local ISPs and Tier-3 ISPs which are connected to the end users. These network operators are customers of the higher Tier ISPs and pays transit fees to connect to the rest of the Internet. In Figure \ref{fig:interArchiOld}, it can be observed that the content hosted in Access ISP A must be passed through different level of hierarchical structure to reach Access ISP D. The content from Access ISP A has to transmit up first to Tier-2 A network operator which forwards content to Tier-1 A operator. Tier-1 A operator and Tier-1 B operator peer with each operator. Tier-1 B operator then transmit it down to Tier-2 C operator which forwards it to Access ISP D operator. Hence the content has to transmit up and down to reach the destination.

\begin{figure}[htb]
  \centering
  \includegraphics[width=14cm]{/home/sakib/soumya/latexNew/latex/archiNew.png}\\
  \caption{Modern Internet Structure}
  \label{fig:interArchiNew}
\end{figure}

\clearpage

\noindent But with the time, the Internet architecture has changed. Researchers found that now nobody has control over Internet, instead each ISP has control over its network and depend upon the network connected with it. Even during last decade the old pyramidal structure of Internet architecture shown in Figure \ref{fig:interArchiOld}, has been bypassed by big content providers, such as Google, Facebook, Amazon or Yahoo!, and content delivery network operators, such as Akamai. As a result now Internet's backbone has a flatter structure where there are few autonomous systems are playing major role in delivering content. They are connected to each other and have a big footprint by establishing small data centers all over the world. This helps them to get as close as possible to the access networks used by their customers, bypassing intermediate Internet service providers. The trend towards flatter network architectures can also be found in the area of access networks. In Figure \ref{fig:interArchiNew}, it can be observed that content hosted in ISP1 “bypass the network” and easily reach the end users thanks to direct and faster connections. This bypass means improved in network performance in delivering content to the end user and, at the same time,  it optimizes network resources by using cache servers near to the end user. The researchers termed these infrastructure providers as “hyper giants” which include large content providers, such as Google and Yahoo, as well as highly distributed CDNs, like Akamai. Hyper giants construct peering links with different autonomous systems (ASs). In this way, they are able to reduce the transit cost of traffic traversing through large ISPs and able to serve content in a faster way.

\subsection{Domain name System (DNS)}
\noindent Domain name system (DNS) is used to translate IP address to corresponding
host names. Internally it is maintaining a hierarchal structure of domains. The administration of domains is divided into different zones. The zone information is distributed using authoritative name servers. The top most level of DNS starts with root zone. The root zone of the DNS
system is centrally administered and serves its zone information via a collection of
root servers. The root servers delegates responsibility for specific zones to some other authoritative name servers which in turn divided responsibility with other authoritative name server. At the end, each site is responsible for its own zone and keep maintain its own database of authoritative name server. The information about a particular domain of a zone is kept in Resource Records (RRs) which specify the class and type of the record as well as the data describing it. Multiple RRs with the same name, type and class are called a resource record set (RRset) \cite {Frank} \cite{Mike4}.\\

\noindent To resolve a host name to IP address, the procedure starts with the end
user’s stub resolver queries to local name server called caching server. If the caching
server can not able to resolve it, it redirects the query to authoritative name server
of the domain. If resolver does not know how to contact the corresponding authoritative name server of the domain, it redirects the query to root name server. The root name server again refers the resolver to the authoritative name server responsible for the domain just below the root server. This procedure continues till resolver is able to resolve the domain properly.\\

\noindent A top-level domain (TLD) is one of the domains at the highest level in the hierarchical Domain Name System of the Internet \cite{dns}. The top-level domain names are installed in the root zone of the name space. For all domains in lower levels, it is the last part of the domain name, that is, the last label of a fully qualified domain name. For example, in the domain name www.example.com, the top-level domain is \enquote{.com}. In the Domain Name System (DNS) hierarchy, a second-level domain (SLD) is a domain that is directly below a top-level domain (TLD). Second level domain normally refers to the organization name that registered the domain name with a domain name registrar.\\

\noindent Listing \ref{lst:listing1} shows the DNS reply by the resolver when querying 
the host name \enquote{www.bmw.com}.\\\\

\begin{lstlisting}[caption= DNS Reply for a host using dig command line tool, label=lst:listing1]

; <<>> DiG 9.10.3-P4-Ubuntu <<>> www.bmw.com
;; global options: +cmd
;; Got answer:
;; ->>HEADER<<- opcode: QUERY, status: NOERROR, id: 60134
;; flags: qr rd ra; QUERY: 1, ANSWER: 4, AUTHORITY: 0, ADDITIONAL: 1

;; OPT PSEUDOSECTION:
; EDNS: version: 0, flags:; udp: 512
;; QUESTION SECTION:
;www.bmw.com.			IN	A

;; ANSWER SECTION:
www.bmw.com.		3600	IN	CNAME	cn-www.bmw.com.edgesuite.net.
cn-www.bmw.com.edgesuite.net. 3600 IN	CNAME	a1586.b.akamai.net.
a1586.b.akamai.net.	20	IN	A	104.121.76.64
a1586.b.akamai.net.	20	IN	A	104.121.76.49

;; Query time: 22 msec
;; SERVER: 127.0.1.1#53(127.0.1.1)
;; WHEN: Fri Oct 14 09:46:54 CEST 2016
;; MSG SIZE  rcvd: 143
\end{lstlisting}



\noindent There are many types of resource records, the most common being A (which gives an IPv4 address for a host name), AAAA (which gives an IPv6 address), MX (which sets the location of a mail server), CNAME (or canonical name, which maps one NAME to another), and TXT (which can include any arbitrary text). In Listing \ref{lst:listing1}, it can be observed that, the host name is \enquote{www.bmw.com}. The answer section contain a chain of CNAMEs \enquote{cn-www.bmw.com.edgesuite.net} and \enquote{a1586.b.akamai.net} which resolve into same ARecord set (RRset) with different IP addresses 104.121.76.64 and 104.121.76.49. The host infrastructure correspond to final IP address is normally refer to as ARecord name. In Listing \ref{lst:listing1}, \enquote{a1586.b.akamai.net} is the ARecord name for both the IP addresses  104.121.76.64 and  104.121.76.49.\\

\subsection{Hyper text transfer protocol(HTTP)}
\noindent Hyper text transfer protocol (HTTP) \cite{RISTOL} \cite{RISTOL1} is an application layer protocol mainly used as defector standard to transport content in world wide web. HTTP works on top
of the TCP/IP protocol and follows the client server architecture via request-response communication procedure. It allows end-users to request, modify, add or delete resources identified by uniform resource identifiers (URIs).\\

\noindent HTTP message consists of HTTP header which shows the meaning of message and HTTP body which is actual message. HTTP message can be a request message or response message. Both types of message consist of a start-line, zero or more header fields (also known as "headers"), an empty line (i.e., a line with nothing preceding the CRLF) indicating the end of the header fields, and possibly a message-body.\\

\noindent The HTTP client sends a request message to server. There are different types of methods used in HTTP request message like GET, HEAD, POST , PUT ,DELETE, TRACE etc. But in this thesis we have used extensible GET method and the HEAD method. HTTP defines these methods to indicate the desired action to be performed on the identified resource.The GET method is used to retrieve information
from the given server using a given URI. Requests using GET should only retrieve data and should have no other effect on the data. The HEAD method request for a response identical to that of a GET request but without the response body.The head method transfers the status line and the header section only. The POST method requests the origin server accept the entity enclosed in the request as a new subordinate of the resource identified by the Request-URI in the Request-Line. The actual function performed by the POST method is determined by the server and is usually dependent on the Request-URI. The PUT method requests that the enclosed entity be stored under the supplied Request-URI. If Request-URI refers to an existing resource, then the enclosed entity is considered as the modified resource. The DELETE method requests that the origin server delete the resource identified by the Request-URI. TRACE allows the client to see what is being received at the other end of the request chain and use that data for testing or diagnostic information.\\\

\begin{lstlisting}[caption= HTTP Get Request message, label=lst:Get]
GET / HTTP/1.1
Host: www.example.com
User-Agent: Mozilla/5.0 [...]
Accept: text/html [...]
Accept-Language: en-us
Accept-Encoding: gzip,deflate
Connection: Keep-alive
\end{lstlisting}

\noindent Listing \ref{lst:Get} shows HTTP Get request message sample for host \enquote{www.example.com}. The introductory line in an HTTP request message consists of method, a server-side path, and the HTTP version in use. \\\

\begin{lstlisting}[caption= HTTP Response message, label=lst:response]
HTTP/1.1 200 OK
Accept-Ranges: bytes
Content-Type: text/html
Date: Mon, 27 Jul 2009 12:28:53 GMT
Server: Apache/2.2.14 (Win32)
Last-Modified: Wed,22 Jul 2009 19:15:56 GMT
Content-Length: 88
<!doctype html>
<html>
[...] 
\end{lstlisting}

\noindent Listing \ref{lst:response} shows HTTP response message sample for host \enquote{www.example.com}. The introductory line in an HTTP response starts out with the HTTP version in use, followed by a standardized three-digit status code and a textual status description. The status code tells the requester about the success of the query or indicates the reason of an error. \\

\noindent Both request and response messages are followed by multiple header lines. Some header information are valid for request, some are for response and some are valid in both the ways. Since HTTP1.1, the host header is mandatory for request messages. The meta information encompasses information about the file type, the character set in use, preferred language etc. HTTP also allows server to set cookies in client side which help the server to track client requests.\\

\subsection{Conclusion}
\noindent Within last decade the Internet architecture changed vastly due to the introduction of hyper giants which can be highly distributed CDNs, cloud computing
CDNs etc. Todays Internet traffic is dominated by HTTP traffic. Furthermore, to deliver
the content fast, DNS protocol is used as the load balancing mechanism by these
big hyper giants. DNS  acts as entry point for the end users to connect to hosting infrastructure. Hence it is ideal to carry out the study to analyze the hyper giants using the DNS.
\clearpage