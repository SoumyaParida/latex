With the proliferation of the Internet,hyper giants such as Google ,Content delivery networks like Akamai etc. often play a vital role in providing the content of any website. These hyper giants not only provide different services but also rich in content. Flash medias from Youtube, login system from Google,Facebook, advertisements from Google ad sense,popular social media sites like Facebook,Twitter,LinkedIn etc. are common and popular services embedded in most of the websites today. 

To cope with this demand, hyper giants have been deploying a large number of scalable and cost effective hosting and content delivery infrastructures all around the globe. These hosting infrastructures can be composed of a few large data centers ,a large number of caches or any combination. Such a scenario cause a large amount of traffic flow from hyper giants as well as huge dependency between popular websites and hyper giants.

In order to know how the involvement of popular websites with hyper giants is evolving,this thesis addresses the following research questions .Firstly,are there any presence of these hyper giants in Internet architecture ? If there are hyper giants present,then what percentage of popular web sites out of 100,000 top websites of Alexa are connected with various hyper giants ? Thirdly ,What percentage of different objects (text/html ,img ,script,media etc.) are connected with hyper giants ?

This thesis provides a quantitative research of web connectivity for Alexa's top 100,000 websites. We present the design, implementation and analyses of different types of objects contained in various websites which are connected to different hyper giants . The thesis follows two paths. Firstly ,in order to quantify different types of objects involved in websites ,we will crawl home pages of top 100,000 websites of Alexa's website and gather the presence of hyper giants infrastructure linked to different objects like images,external links,scripts ,embedded videos,css files etc. in those websites. Secondly, We examine those objects to find out the degree of connection between the top 100,000 websites and hyper giant.

The experimental results discussed in this thesis are supported by extensive analyses of data collected which provide evidence in support of the conclusion provided in this thesis.
