In a relatively short period of time, the Internet has an amazing impact on almost every facet of our lives. With it, we are able to access new ideas, more information, unlimited possibilities, and a whole new world of communities. In addition, the demand for more and richer content has increased. To fulfill this demand, the Internet has evolved immensely in last decade. Content become the king. Websites are becoming very rich in content as well as delivering high quality content to their customers. On top, introduction of social networking platforms such as Facebook, Twitter; video sharing sites such as Youtube has changed the user interaction with Internet by enabling them to publish their own content and share with each other which led to an exponential growth of Internet traffic. 

To adopt this new trend, some companies are following state-of-the-art strategies for distributing their content while offering the best user experience. They have been deploying a large number of scalable and cost effective hosting and content delivery infrastructures all around the globe. These hosting infrastructures can be composed of a few large data centers ,a large number of caches or any combination. These companies are termed as hyper giants in today's Internet. Other websites use the infrastructure of these hyper giants to deliver content. Such a scenario cause a large amount of traffic flow from hyper giants as well as creating huge dependency between other websites and hyper giants.

In order to know how the involvement of popular websites with hyper giants is evolving,this thesis addresses some of the common research questions. Firstly, whether there any presence of these hyper giants in today's Internet? Secondly, how the inter dependency between hyper giants and popular websites are evolving with change in Internet trend. Being able to identify such companies, is helpful not only to other content distributing companies, content producers, content providers, and ISPs, but also to the research community at large. 

In this thesis, we purposes a mechanism to identify the hyper giants and their inter dependency with popular web sites. The thesis is conducted on a single vantage point in Germany. The experimental results discussed in this thesis are supported by extensive analyses of data collected which provide evidence in support of the conclusion provided in this thesis.
