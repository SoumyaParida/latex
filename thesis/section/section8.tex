There are certain areas which can be investigated further in future which are not covered in the current scope of this thesis.This section will discuss about these points.

\begin{itemize}
\item First of all the thesis is done at single vantage point at Germany.Hence the identification of hyper giants,their role and their relationship with popular websites can be changed when the whole procedure will be done in whole world basis.Hence it will be interesting to see how the clustering algorithm works when taking all the SLDs across the world.

\item Secondly while clustering the hyper giants,the k-means parameter is taken by going through very small number of observations.Hence in future this can be tested more precisely which will help to cluster the hosting infrastructures at a granular level.

\item In the clustering algorithm,we club two SLDs if one SLD prefixes matches with other SLD prefixes by more than equal to 70\%.This matching index is chosen after extensively testing.Hence this can be taken as future work to see what is the best matching index.

\item After clustering algorithm,we have chosen to take  top 5.95 \% (=3205) clustered SLD infrastructures which serve almost 80\% of links and have 78.65 \% of SLDs .In this way we argumented to get most of the big hosting infrastructures as well as content providers for further analysis.Hence this can be taken for future work to validate the argument properly.

\end{itemize}